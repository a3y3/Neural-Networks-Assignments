\documentclass[12pt]{article}
\usepackage{fullpage,enumerate,amsmath,amssymb,graphicx}
\begin{document}

\begin{center}
{\Large ICT 4012 Spring 2017 Homework [1]} %For example: Homework 1

\begin{tabular}{rl}
MU Registration No.: & [140911090] \\  % Enter your MU Registration 
Name: & [Soham Dongargaonkar] \\   % Enter your full name
%Collaborators: & [list all the people you worked with (if permitted)] % Only if permitted by the teacher, other just comment this line
\end{tabular}
\end{center}

%==============Do not change this statement===================================================================
By turning in this assignment, I agree by the academic honor code and declare that all of this is my own work.
%=============================================================================================================

\section*{Problem 1}

\begin{enumerate}
    \item \begin{gather}
                \phi(a) =  \frac{1}{1+e^{-av}} \nonumber\\ \nonumber\\
                \frac{d}{dv}\phi(a) = \frac{d}{dv}(\frac{1}{1+e^{-av}})\nonumber\\\nonumber\\
                \frac{d}{dv}\phi(a) = \frac{ae^{av}}{(1+e^{-av})^2} \label{eqn:1}
            \end{gather}
            \[ 
                \text{[Substituting\ v = 0]} 
            \]
             
            \begin{gather}
                \phi'(v) = \frac{a}{4}
            \end{gather}  
               

   
\end{enumerate}

\section*{Problem 2}

\begin{enumerate}
  \item \[
            \text{In this question, we must vary a and see how the function changes values.}
        \]
        \[
            \text{Hence, a is not a constant anymore. We assume v as constant in this proof.}
        \]
        \begin{gather}
            \therefore\phi(a) = \tanh(av) \nonumber \\ \nonumber\\ 
            \lim_{a\to\infty} \phi(a) = \lim_{a\to\infty}\tanh(av) \label{eqn:2}
        \end{gather}
        
        \[
            \text{But\ $tanh(av)$ =$\frac{e^{av}-e^{-av}}{e^{av}+e^{-av}}$} \\
        \]
        
        \begin{gather}             
            \therefore \lim_{a\to\infty} \phi(a) = \lim_{a\to\infty}\frac{e^{av}-e^{-av}}{e^{av}+e^{-av}} \label{eqn:3}    
        \end{gather}
        \[
            \text{For $\lim_{a\to\infty}$, take $\frac{e^{av}-e^{-av}}{e^{av}+e^{-av}}$=$\frac{1-e^{-2ah}}{1+e^{-2ah}}$}
        \]
        \[
            \text{And for $\lim_{a\to-\infty}$, take $\frac{e^{av}-e^{-av}}{e^{av}+e^{-av}}$=$\frac{e^{2ah}-1}{e^{2ah}+1}$}
        \]
        
        \[
            \text{Applying appropriate limits, for $\lim_{a\to\infty}$, we get} 
        \]
        \begin{gather}
            \frac{1-0}{1+0} = 1
        \end{gather}
        
        \[
            \text{And for $\lim_{a\to-\infty}$, we get} 
        \]
        \begin{gather}
            \frac{0-1}{0+1} = -1
        \end{gather}
       
       \[
            \text{Hence, } 
        \]
\end{enumerate}


\end{document}
